Blog Post \ref{whyq} introduced the concept of evenly spread points, which are commonly referred to as \emph{low discrepancy} points.  This is in contrast to independent and identically distributed (IID) points.  

Consider two sequences,
\begin{equation}
    \vX_1, \vX_2, \ldots \overset{\text{IID}}{\sim} \cu[0,1]^d, \qquad \vT_1, \vT_2, \ldots \overset{\text{LD}}{\sim} \cu[0,1]^d.
\end{equation}
Both sequences are expected to look like points spread uniformly over the unit cube, $[0,1]^d$.  
The first sequence must be random, or as random looking as our deterministic random number generators can make it.  Since the points are independent, the location of any $\vX_i$ should have no bearing on the location of any other $\vX_j$.  Removing a point at random does not affect the IID property.

The second sequence may be random or deterministic.  Let $F_{\{\vT_1, \ldots, \vT_n\}}$ denote the empirical distribution function of the first $n$ points of this sequence, i.e., the probability distribution that assigns a probability $1/n$ to each location $\vT_i$.  For $\vT_1, \vT_2, \ldots$ to be a low discrepancy sequence, $F_{\{\vT_1, \ldots, \vT_n\}}$ should be close to the uniform probability distribution, $F_{\text{unif}}(\vx) = x_1 \cdots x_d$.  ``Close'' implies that we can measure how far apart two distributions are, and we call this measure the \emph{discrepancy}.

Just like beauty is in the eye of the beholder, so there are different  

\cite{Hic97a}