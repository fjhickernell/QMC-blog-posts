\documentclass{article}
\usepackage[utf8]{inputenc}
\usepackage{amsmath,
	amssymb,
	amsthm,
	datetime,
	hyperref,
	cleveref,
	float,
	mathtools,
	bbm,
	%mathabx,
	array,
	booktabs,
	xspace,
	calc,
	colortbl,
	siunitx,
	url,
	xspace,
 	graphicx}
\usepackage{hyperref}
\hypersetup{
    colorlinks=true,
    linkcolor=blue,
    filecolor=magenta,      
    urlcolor=cyan,
}
\usepackage[dvipsnames]{xcolor}
\usepackage{listings}
\definecolor{darkgreen}{rgb}{0,0.6,0}
\lstdefinestyle{Python}{
    language        = Python,
    basicstyle      = \ttfamily,
    keywordstyle    = \color{blue},
    keywordstyle    = [2] \color{teal}, % just to check that it works
    stringstyle     = \color{green},
    commentstyle    = \color{darkgreen}\ttfamily
}

\usepackage[giveninits=false,backend=biber,style=nature, maxcitenames =10, mincitenames=9]{biblatex}
\newcommand{\HickernellFJ}{Hickernell\xspace}
\addbibresource{main.bib}
\addbibresource{FJHown23.bib}
\addbibresource{FJH23.bib}
\title{QMC Blog Posts}
\author{QMCPy Crew}
\date{May 2020}
\input FJHDef.tex

\newcommand{\blogpost}[4]{\newpage%
\section{#1}%
\begin{refsection}%
    \label{#3}%
	Author(s): #2 \newline \newline%
	\input{#4.tex}%
\printbibliography[heading=subbibliography]
\end{refsection}
} %This command provides a new section and bibliography for each blogpost

\newcommand{\myshade}{60}
\colorlet{mylinkcolor}{violet}
\colorlet{mycitecolor}{violet}
\colorlet{myurlcolor}{YellowOrange}

\hypersetup{
	linkcolor  = mylinkcolor!\myshade!black,
	citecolor  = mycitecolor!\myshade!black,
	urlcolor   = myurlcolor!\myshade!black,
	colorlinks = true,
}

\newcommand{\FredComment}[1]{{\color{blue} Fred: #1}}
\newcommand{\AGSComment}[1]{{\color{green} Aleksei: #1}}
\newcommand{\SCComment}[1]{{\color{red} Sou-Cheng: #1}}
\newcommand{\MikeComment}[1]{{\color{orange} Mike: #1}}
\newcommand{\JagsComment}[1]{{\color{violet} Jags: #1}}

\theoremstyle{definition}
\newtheorem{example}{Example}[section]

\begin{document}

\maketitle

\setcounter{tocdepth}{1}
\tableofcontents

\section*{Introduction}

We will be using this to construct and edit blog content for the launch of QMCPy.

Fred prefers to have each blog focus on just one idea, short and sweet.

The Google doc where we are defining the launch concept around the blogs is here \url{https://docs.google.com/document/d/1DcDdmwdUDCNKxGwUWCyQUjkTuQjhn-dskajlnEq14bo/edit?usp=sharing}

\blogpost{Why Add Q to MC?}{Fred Hickernell}{whyq}{WhyQ/WhyQ}

\blogpost{A QMCPy Quick Start}{Sou-Cheng Choi and Aleksei Sorokin}{quickstart}{Quickstart/Quickstart}

\blogpost{What Makes a Sequence ``Low Discrepancy''?}{Fred Hickernell}{lds}{LDS/LDS}

\blogpost{A Brief History of Quasi-Monte Carlo}{}{}{}

\blogpost{The People of Quasi-Monte Carlo}{}{}{}

\blogpost{Be Careful Replacing IID Sequences by Low Discrepancy Sequences}{}{}{}

\blogpost{Randomizing Your Low Discrepancy Points Helps}{}{}{}

\blogpost{What to Do when the Dimension is Infinite}{}{}{}

\blogpost{Variation/Variance Reduction for QMC}{}{}{}

\blogpost{When QMC Does Not Help}{}{}{}


\end{document}
