\documentclass{article}

% reduce margins 
\usepackage[papersize={8.5in,11in}]{geometry}

\usepackage[utf8]{inputenc}
\usepackage{verbatim}
\usepackage{amsthm}
\usepackage[most]{tcolorbox}
\usepackage[dvipsnames]{xcolor}
\usepackage{amsmath,
    amssymb,
    datetime,
    hyperref,
    cleveref,
    float,
    mathtools,
    bbm,
    %mathabx,
    array,
    booktabs,
    xspace,
    calc,
    colortbl,
    siunitx,
    url,
    xspace,
    graphicx,
    bigints,
    textcomp}

\usepackage{hyperref}
\hypersetup{
    colorlinks=true,
    linkcolor=blue,
    filecolor=magenta,      
    urlcolor=blue,
}
\usepackage[dvipsnames]{xcolor}
\usepackage{listings}

% prevents long code lines in listings overflow
\lstset{
  basicstyle=\fontfamily{lmvtt}\selectfont\small\color{blue},
  columns=fullflexible,
}

\definecolor{darkgreen}{rgb}{0,0.6,0}
\lstdefinestyle{Python}{
    language        = Python,
    basicstyle      = \ttfamily,
    keywordstyle    = \color{blue},
    keywordstyle    = [2] \color{teal}, % just to check that it works
    stringstyle     = \color{green},
    commentstyle    = \color{darkgreen}\ttfamily,
    frame=single,
    numbers=left,
    upquote=true % straight quote; need textcop package
}
\definecolor{mGreen}{rgb}{0,0.6,0}
\definecolor{mGray}{rgb}{0.5,0.5,0.5}
\definecolor{mPurple}{rgb}{0.58,0,0.82}
\definecolor{backgroundColour}{rgb}{0.95,0.95,0.92}

\lstdefinestyle{CStyle}{
    backgroundcolor=\color{backgroundColour},   
    commentstyle=\color{mGreen},
    keywordstyle=\color{magenta},
    numberstyle=\tiny\color{mGray},
    stringstyle=\color{mPurple},
    basicstyle=\footnotesize,
    breakatwhitespace=false,         
    breaklines=true,                 
    captionpos=b,                    
    keepspaces=true,                 
    numbers=left,                    
    numbersep=5pt,                  
    showspaces=false,                
    showstringspaces=false,
    showtabs=false,                  
    tabsize=2,
    language=C
}
\usepackage[giveninits=false,backend=biber,style=nature, maxcitenames =10, mincitenames=9]{biblatex}
\newcommand{\HickernellFJ}{Hickernell\xspace}
\addbibresource{main.bib}
\addbibresource{FJHown23.bib}
\addbibresource{FJH23.bib}
\title{QMC Blog Posts}
\author{QMCPy Crew}
\date{May 2020}
\input FJHDef.tex

\newcommand{\blogpost}[5][{}]{\newpage%
\section{#2 {\color{teal}#1}}%  #1 is publication date, #2 is title
\begin{refsection}%
    \label{#4}%  #4 is label of section
	Author(s): #3 \newline \newline% #3 is authors
	\input{#5.tex}%. #5 is location of the file with the text
\printbibliography[heading=subbibliography]
\end{refsection}
} %This command provides a new section and bibliography for each blogpost

\newcommand{\myshade}{60}
\colorlet{mylinkcolor}{violet}
\colorlet{mycitecolor}{violet}
\colorlet{myurlcolor}{YellowOrange}

\begin{comment}
\hypersetup{
	linkcolor  = mylinkcolor!\myshade!black,
	citecolor  = mycitecolor!\myshade!black,
	urlcolor   = myurlcolor!\myshade!black,
	colorlinks = true,
}
\end{comment}

\renewcommand{\vtheta}{{\bvec{\theta}}}
\newcommand{\err}{{\textup{err}}}
\newcommand{\CI}{{\textup{CI}}}
\newcommand{\MLE}{{\textup{EB}}}
\newcommand{\code}[1]{{\texttt{#1}}}
\newcommand{\bm}[1]{\boldsymbol{#1}}
\newcommand{\dvt}{\dif{\bm{t}}}

\newcommand{\FredComment}[1]{{\color{blue} Fred: #1}}
\newcommand{\AGSComment}[1]{{\color{cyan} Aleksei: #1}}
\newcommand{\SCComment}[1]{{\color{red} Sou-Cheng: #1}}
\newcommand{\MikeComment}[1]{{\color{orange} Mike: #1}}
\newcommand{\JagsComment}[1]{{\color{violet} Jags: #1}}
\DeclareMathOperator{\disc}{disc}

\newcommand{\R}{{\mathbb{R}}} % reals


\theoremstyle{definition}
\newtheorem{example}{Example}[section]



\begin{document}

\maketitle

\setcounter{tocdepth}{1}
\tableofcontents

\section*{Introduction}

We will be using this to construct and edit blog content for the launch of QMCPy.

Fred prefers to have each blog focus on just one idea, short and sweet.

The Google doc where we are defining the launch concept around the blogs is here \url{https://docs.google.com/document/d/1DcDdmwdUDCNKxGwUWCyQUjkTuQjhn-dskajlnEq14bo/edit?usp=sharing}

% \blogpost[published June 25, 2020]{Why Add Q to MC?}{Fred Hickernell}{whyq}{WhyQ/WhyQ}

% \blogpost[published July 6, 2020]{A QMCPy Quick Start}{Sou-Cheng Choi and Aleksei Sorokin}{quickstart}{Quickstart/Quickstart}

% \blogpost[published July 8, 2020]{What Makes a Sequence ``Low Discrepancy''?}{Fred Hickernell}{lds}{LDS/LDS}

% \blogpost[published Feb 25, 2021]{Speeding up QMCPy with Distributable C Code}{Aleksei Sorokin and Jagadeeswaran Rathinavel}{clib}{Clib/clib}

% \blogpost[published Feb 25, 2021]{Visualizing the Internals of Object Classes in QMCPy}{Aleksei Sorokin and Sou-Cheng Choi}{uml}{uml/uml}

% \blogpost{Digital Sequences in QMCPy}{Aleksei Sorokin}{sobol}{digital_seq/digital_seq}

% \blogpost{Samples Sizes for LD Sequences}{}{}{}

% \blogpost{Bayesian Stopping Criteria}{Jagadeeswaran R, Fred J. Hickernell}{BayesCub}{BayesCub/bayesianCubature}

% \blogpost{Validating the Gaussian assumption}{Fred J. Hickernell, Jagadeeswaran R}{BayesCub}{BayesCub/guassianDiagnostics}

% \blogpost{Methods to optimize rare-event Monte Carlo reliability simulations for Large Hadron Collider Protection Systems}{Milosz Robert Blaszkiewicz}{hadron}{hadron_collider_protection/main}

% \blogpost{QMC When Answers are Vectors}{}{}{}

% \blogpost{Random Lattice Generators Are Not Bad}{David Zhang}{randlattice}{TotallyRandomLattice/totallyrandomlattice}

 \blogpost{Netting the Digital Chaos: Exploring Totally Randomized Digital Nets}{David Zhang}{}{TotallyRandomDnet/totallyrandomdnet}

% \blogpost{Be Careful Replacing IID Sequences by Low Discrepancy Sequences}{Fred Hickernell}{}{}

% \blogpost{Randomizing Your Low Discrepancy Points Helps}{Fred Hickernell}{}{}

% \blogpost{What to Do when the Dimension is Infinite}{}{}{}

% \blogpost{Variation/Variance Reduction for QMC}{}{}{}

% \blogpost{When QMC Does Not Help}{}{}{}

% \blogpost{Math 565 Final project posts}{}{}{}

% \blogpost{The People of Quasi-Monte Carlo}{}{}{}

% \blogpost{The Advantage of Low Discrepancy Over Latin Hypercube Sampling}{}{}{}

% \blogpost{Historical Highlights for Quasi-Monte Carlo}{Fred Hickernell}{history}{History/History}

\blogpost{Visualizing the Generated Samples Helps}{Aadit Jain}{plotproj}{PlotProjectionsFunction/plotprojectionsfunction}

\blogpost{CubMCCLTVec: Vectorizing the CubMCCLT Algorithm}{Aadit Jain}{cubmccltvec}{CubMCCLTVec/cubmccltvec}

\blogpost{Linear Matrix Scrambling and Digital Shift for Halton}{Aadit Jain}{lms_ds_halton}{LMS_DS_Halton/lms_ds_halton}

%\blogpost{Introducing Geometric Brownian Motion in QMCPy}{Larysa}{gbm}{GBM/gbm}

\blogpost{Parsl Accelerated QMCpy Notebook Tests}{Dr. Sou-Cheng Choi and Joshua Jay Herman (QMC Development Team)}{parslTestbookQMCblog}{booktests/parslTestbookQMCblog}
\end{document}
