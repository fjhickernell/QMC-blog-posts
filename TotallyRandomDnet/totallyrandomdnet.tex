Digital nets offer a framework for constructing low-discrepancy sequences that are particularly useful for high-dimensional integration. The primary advantage of using digital nets over traditional Monte Carlo sampling is their superior coverage of the integration domain. Because of this, they can often yield more accurate estimates with fewer samples. Randomized Digital Nets are a key component of randomized Quasi Monte-Carlo (RQMC) methods, which tend to behave more robustly than their unrandomized counterparts. Additionally, recent works by Zexin Pan and Art Owen \cite{panowen2022superpolynomial} studied the effects of the median-of-means sampling method on digital nets and derived fruitful results. 

Generating matrices stands as a foundational pillar for the construction of digital nets. These matrices, defined by specific mathematical criteria, ensure the low-discrepancy properties of the nets, optimizing them for high-dimensional numerical integration and other computational challenges. Analogous to totally randomized lattices based on randomized generating vectors \cite{davidblogqtomc2023}, we recently implemented totally randomized generating matrices in QMCPy. This blog discusses both the mathematics and code usage of randomized digital nets in QMCPy. 

\section*{Mathematics of Digital Nets}

For a given dimension \( s \geq 1 \), an integer base \( b \geq 2 \), a positive integer \( m \), and an integer \( t \) with \( 0 \leq t \leq m \), a point set \( \mathcal{P} \) of \( b^m \) points in \( [0,1]^s \) is called a \( (t,m,s) \)-net in base \( b \). The property for \(\mathcal{P}\) to be a \( (t,m,s) \)-net in base \( b \) means that every interval \( J = \prod_{i=1}^s \left[ \frac{A_i}{b^{d_i}}, \frac{A_i + 1}{b^{d_i}} \right] \) with \( d_1 + \ldots + d_s = m - t \), that is, of volume \( b^{-m+t} \), contains exactly \( b^t \) points of \( \mathcal{P} \) \cite{Dick_Pillichshammer_2010}.

In other words, a net is a point set for which a partition of the $s$ dimension cube with equal volume that contains an equal amount of points always exists. 

%insert a graph of a digital net, highlighted, perhaps generated by QMCPy

To learn more about digital nets, please visit \href{/https://web.maths.unsw.edu.au/~josefdick/preprints/DP_book_preprint.pdf}{DigitalNetsandSequences}
%link does not work

The term \emph{digital nets} refers to a construction method based on digital operations within finite fields or rings. Such nets are constructed using linear algebraic techniques over finite fields, with representations suitable for digital computation. The construction typically involves the following steps:

\begin{enumerate}
    \item Selection of a base \( b \) and construction of matrices over the finite field \( \mathbb{F}_b \).
    \item Multiplication of these matrices with vectors of digits representing integers in base \( b \).
    \item Normalization of the product to ensure that the resulting points lie within the unit cube \( [0,1]^s \).
\end{enumerate}

The term \emph{digital} emphasizes the use of discrete operations—akin to those performed by digital computers—which manipulate bits of information. Harald Niederreiter introduced this concept, and it is a key component in quasi-Monte Carlo methods, where digital nets are used to create low-discrepancy sequences \cite{niederreiter1992RNGQMC}. These sequences aim to cover the multidimensional space more uniformly than random points, which is advantageous in high-dimensional numerical integration tasks.

\section*{QMC Implementation}

The class \code{DigitalNetB2} in QMCPy provides an implementation for generating digital nets. As the name suggests, all nets generated follow a $b=2$ generating scheme. The current generation procedure takes an input integer $m \ge 2$ and dimension $s \ge 1$, setting the parameter $m$. Subsequently, for each dimension, a sequence of length $m$ starts with 1 (index $i=0$), followed by random integers between $[2^i,2^i+1)$. These numbers are then transcribed into their binary representations and aligned in a matrix $M$, an upper-triangular $m\times m$ matrix with $1$ on the diagonal and the other elements randomized between $1$ and $0$. 

For example, consider $m = 4$ and $s = 1$, and the sequence $1, 2, 6, 11$. The binary representation of these numbers is $0001, 0010,0110,1011$. Stacking these numbers vertically from left to right in a matrix yields
\[
\begin{bmatrix}
1 & 0 & 0 & 1\\
0 & 1 & 1 & 1\\
0 & 0 & 1 & 0\\
0 & 0 & 0 & 1
\end{bmatrix}	
\]





Currently, we only support up to 64 dimensions of generating vectors




To this end, suggested in their work \cite{doi:10.1137/22M1473625} that generating vectors do not require a predetermined weight vector and a  measure of the decay of Fourier coefficients to reach a high precision; instead, random generators can also achieve desirable results.


