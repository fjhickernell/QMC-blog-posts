Quasi-Monte Carlo methods got their start in the late 1950s, not long after Monte Carlo methods became popular through the work of Stan Ulam, John von Neumann, and their collaborators \cite{Eck87a}.  Robert Richtmeyer, a colleague of von Neumann's at Los Alamos National Laboratory introduced a deterministic sequence of numbers that we would call low discrepancy and coined the term ``Quasi-Monte Carlo''\cite{Ric51}.  

\paragraph{Discrepancy.}

\paragraph{Low Discrepancy Sequences.} John Halton proposed a low discrepancy sequence that bears his name and can be found in several software packages \cite{Hal60}.

\paragraph{High and Infinite Dimensional Problems.}

\paragraph{Uncertainty Quantification.}