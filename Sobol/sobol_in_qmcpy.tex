The Sobol' sequence is one of the most popular and widely used low-discrepancy sequences. QMCPy's Sobol'generator builds upon existing generators to provide developers and practitioners advanced, customized, usage.

Existing Sobol' implementations include work from  Brantly and Fox [8], the Quasi-Random Number Generator project [1,2], and the Magic Point Shop library [3,4]. Popular python packages such as SciPy and PyTorch have also been working on their own Sobol' implementations. 

So why use QMCPy's Sobol' generator? The Sobol' generator in QMCPy allows for customization such as multiple randomization methods, custom dimensions, vectorized seeding, and custom generating vectors. 




\subsection{References}
[1] Marius Hofert and Christiane Lemieux (2019). qrng: (Randomized) Quasi-Random Number Generators. R package version 0.0-7. https://CRAN.R-project.org/package=qrng.

[2] Faure, Henri, and Christiane Lemieux. “Implementation of Irreducible Sobol' Sequences in Prime Power Bases.” Mathematics and Computers in Simulation 161 (2019): 13–22. Crossref. Web.

[3] F.Y. Kuo & D. Nuyens. Application of quasi-Monte Carlo methods to elliptic PDEs with random diffusion coefficients
- a survey of analysis and implementation, Foundations of Computational Mathematics, 16(6):1631-1696, 2016. springer link: https://link.springer.com/article/10.1007/s10208-016-9329-5 arxiv link: https://arxiv.org/abs/1606.06613

[4] D. Nuyens, `The Magic Point Shop of QMC point generators and generating vectors.` MATLAB and Python software, 2018. Available from
https://people.cs.kuleuven.be/~dirk.nuyens/

[5] Paszke, A., Gross, S., Massa, F., Lerer, A., Bradbury, J., Chanan, G., … Chintala, S. (2019). PyTorch: An Imperative Style, High-Performance Deep Learning Library. In H. Wallach, H. Larochelle, A. Beygelzimer, F. d extquotesingle Alch&#39;e-Buc, E. Fox, & R. Garnett (Eds.), Advances in Neural Information Processing Systems 32 (pp. 8024–8035). Curran Associates, Inc. Retrieved from http://papers.neurips.cc/paper/9015-pytorch-an-imperative-style-high-performance-deep-learning-library.pdf

[6] I.M. Sobol', V.I. Turchaninov, Yu.L. Levitan, B.V. Shukhman: "Quasi-Random Sequence Generators" Keldysh Institute of Applied Mathematics,
Russian Acamdey of Sciences, Moscow (1992).

[7] Sobol, Ilya & Asotsky, Danil & Kreinin, Alexander & Kucherenko, Sergei. (2011). Construction and Comparison of High-Dimensional Sobol' Generators. Wilmott. 2011. 10.1002/wilm.10056.

[8] Paul Bratley and Bennett L. Fox. 1988. Algorithm 659: Implementing Sobol's quasirandom sequence generator.
ACM Trans. Math. Softw. 14, 1 (March 1988), 88–100. DOI:https://doi.org/10.1145/42288.214372