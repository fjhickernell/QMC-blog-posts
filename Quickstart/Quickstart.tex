In this tutorial, we introduce QMCPy \cite{QMCPy2020a_alt}  by example. QMCPy can be installed with \texttt{pip install qmcpy} or cloned from the  \href{https://github.com/QMCSoftware/QMCSoftware/tree/master/python_prototype}{QMCSoftware GitHub repository}.

Consider the problem of integrating the Keister function \cite{Kei96} with respect to a $d$-dimensional Gaussian measure: 
\begin{gather*}
    \text{Keister}(\vx) = \pi^{d/2} \cos(||\vx||), \qquad \vx \in \reals^d, \qquad \vX \sim \mathcal{N}(\vzero_d,\mI_d/2),  \\
    \Ex[\text{Keister}(\vX)] : = \int_{\reals^d} \text{Keister}(\vx) \, \pi^{-d/2} \exp( - ||\vx||^2) \,  \dif \vx = ?,
\end{gather*}
where $||\vx||$ is the \href{https://en.wikipedia.org/wiki/Norm_(mathematics)}{Euclidean norm} and $\mI_d$ is the $d$ dimensional identity matrix. The Keister function is implemented below with help from NumPy \cite{numpy}.

\lstinputlisting[style=Python]{Quickstart/snip1.py}

In addition to our Keister integrand and Gaussian true measure, we must select a discrete distribution and stopping criterion. The stopping criterion determines the number of points at which to evaluate the integrand in order for the mean approximation to be within a specified error tolerance. The discrete distribution determines the sites at which the integrand is evaluated.

For the Keister example we select the lattice sequence as the discrete distribution and corresponding cubature-based stopping criterion. The discrete distribution, true measure, integrand, and stopping criterion are then constructed within the QMCPy framework below. 

\lstinputlisting[style=Python]{Quickstart/snip2.py}

Calling \textit{integrate} on the \textit{stopping\_criterion} returns the numerical solution and a data object. Printing the data object will provide a neat summary of the integration problem.

\lstinputlisting[style=Python]{Quickstart/snip3.py}

This guide is not meant to be exhaustive but rather a quick introduction to the QMCPy framework and syntax. In the following blog we will take a closer look at low-discrepancy sequences such as the lattice sequence from the above example.  
