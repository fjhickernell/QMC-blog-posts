In this tutorial, we introduce QMCPy~\cite{QMCPy2020a}  by example. First, we proposed a mathematical framework for quasi-Monte Carlo problems. Then we develop an example within this framework using the QMCPy package.

\subsection{Mathematical Background}
In general, QMCPy works to approximate the expectation of an integrand $g$ with respect to a true measure~$\rho$ defined over domain $\mathcal{X}$. 

$$\mu = \mathbb{E}(g(\mathbf{x})) = \int_{\mathcal{X}} g(\mathbf{x}) \rho(\mathbf{x}) d\mathbf{x}.$$

QMCPy transforms the true measure $\rho$ to a measure $\nu$ that can be easily sampled from a discrete probability measure $\hat{\nu}$ with domain $\tilde{\mathcal{X}}$. To accommodate the measure transform, the original integrand $g$ is transformed into a new integrand~$f$.

$$\mu = \int_{\tilde{\mathcal{X}}} f(\mathbf{x}) \nu(\mathbf{x}) d\mathbf{x} = \int_{\mathcal{X}} g(\mathbf{x}) \rho(\mathbf{x}) d\mathbf{x}.$$

QMCPy approximates the expectation to be the average of $n$ weighted evaluations of $f$ at samples from $\hat{\nu}$. A stopping criterion (algorithm to determine $n$) is required to approximate the integral to within a tolerance $\epsilon$. 

$$\int_{\mathcal{X}} f(\mathbf{x}) \nu(\mathbf{x}) d\mathbf{x} \approx a_i \sum_{i=1}^n f(\mathbf{x}) w_i = \hat{\mu}_n,$$
$$|\mu-\hat{\mu}_n|<\epsilon.$$

\subsection{Keister Example}
In this example we look at the $d$ dimensional Keister integrand with a $\mathcal{N}_d(0,\frac{1}{2}I)$ true measure, lattice discrete distribution, and lattice cubature stopping criterion. 
$$g(\mathbf{x}) = \pi^{d/2} \cos(||\mathbf{x}||_2) \qquad \text{for} \qquad \mathbf{x} \sim \mathcal{N}_d(0,\frac{1}{2}) = \rho.$$


First we set up the Keister function with help from the numpy package.
\lstinputlisting[style=Python]{Quickstart/snip1.py}

Then, we define our discrete distribution, true measure, integrand, and stopping criterion objects. When the QMCPy object is passed into another objects constructor the transformation from $g$ to $f$ and $\rho$ to $\nu$ are automatically configured. 
\lstinputlisting[style=Python]{Quickstart/snip2.py}

The integrate method of the stopping\_criterion object returns the numerical solution and a data object. Printing the data object will provide a neat summary of the integration problem.
\lstinputlisting[style=Python]{Quickstart/snip3.py}
